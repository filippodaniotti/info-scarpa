\documentclass[aspectratio=1610]{beamer}
\usepackage{style}
\usetheme{madrid}
\usecolortheme{default}
\graphicspath{{./assets/images}}

\title{Il Corso di Laurea in Informatica}
\author{Filippo Daniotti \and Filippo Momesso}
\date{\today}
\logo{\includegraphics[height=0.8cm]{logo-small.jpg}}
\institute[DISI]{Dipartimento di Ingegneria e Scienza dell'Informazione}

% arara: pdflatex: { shell: yes, synctex: yes }
% arara: latexmk: { clean: partial }
\begin{document}
	\begin{frame}[plain]
		% \addtocounter{framenumber}{-1}
		% \begin{figure}
		% 	\centering
		% 	\includegraphics[width=0.5\textwidth, keepaspectratio]{logo-unitn.eps}		
		% \end{figure}
		% \LARGE{\textsc{Dipartimento di Ingegneria e Scienza dell'Informazione}\\}
		% \large{\textsc{DISI}\\}
		% \institute[]{Dipartimento di Ingegneria e Scienza dell'informazione\\}
		\centering
		\includegraphics[width=0.5\textwidth, keepaspectratio]{logo-unitn.eps}		
		\titlepage
	\end{frame}

	\section{Introduzione e formalità}
	\begin{frame}[fragile]{Che cos'è l'informatica?}
		\begin{center}
			\Large{TL;DR}\\
		\end{center}		
		È tante cose, dare una definizione precisa ad alta voce lascia interdetto qualsiasi interlocutore. In un certo senso è quasi più utile parlare di cosa \structure{non} è, questo se non altro ci aiuta a non fare assunzioni sbagliate.\\
		\bigskip
		\pause
		Giusto per puntualizzare quelle che sono di certo ovvietà, l'informatica \structure{non} è:
		\begin{columns}
			\begin{column}{0.55\textwidth}
				\begin{itemize}
					\item l'ECDL 
					\item hackerare i profili social 
					\item preparare le crack per i videogiochi 
					\item riparare i computer \Large{AHHHHHHHH} 
				\end{itemize}		
			\end{column}
			\begin{column}{0.4\textwidth}
				inserire memino qui sul riparare i computer + esprienza 150h
			\end{column}
		\end{columns}
	\end{frame}

	\begin{frame}{Cosa si studia}
		\begin{itemize}
			\item 
		\end{itemize}
	\end{frame}

\end{document}